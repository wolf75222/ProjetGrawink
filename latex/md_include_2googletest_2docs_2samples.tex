\chapter{Googletest Samples}
\hypertarget{md_include_2googletest_2docs_2samples}{}\label{md_include_2googletest_2docs_2samples}\index{Googletest Samples@{Googletest Samples}}
\label{md_include_2googletest_2docs_2samples_autotoc_md372}%
\Hypertarget{md_include_2googletest_2docs_2samples_autotoc_md372}%
 If you\textquotesingle{}re like us, you\textquotesingle{}d like to look at \href{https://github.com/google/googletest/blob/main/googletest/samples}{\texttt{ googletest samples.}} The sample directory has a number of well-\/commented samples showing how to use a variety of googletest features.


\begin{DoxyItemize}
\item Sample \#1 shows the basic steps of using googletest to test C++ functions.
\item Sample \#2 shows a more complex unit test for a class with multiple member functions.
\item Sample \#3 uses a test fixture.
\item Sample \#4 teaches you how to use googletest and {\ttfamily googletest.\+h} together to get the best of both libraries.
\item Sample \#5 puts shared testing logic in a base test fixture, and reuses it in derived fixtures.
\item Sample \#6 demonstrates type-\/parameterized tests.
\item Sample \#7 teaches the basics of value-\/parameterized tests.
\item Sample \#8 shows using {\ttfamily Combine()} in value-\/parameterized tests.
\item Sample \#9 shows use of the listener API to modify Google Test\textquotesingle{}s console output and the use of its reflection API to inspect test results.
\item Sample \#10 shows use of the listener API to implement a primitive memory leak checker. 
\end{DoxyItemize}